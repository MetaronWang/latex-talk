\section{填写创作}



\subsection{文件结构}

\begin{frame}[fragile]{文件结构}
    \lstset{language=[LaTeX]TeX}
    \begin{lstlisting}[basicstyle=\ttfamily]
  \documentclass[degree=doctor,language=chinese,font=external,cjk-font=external]{sustechthesis}
  % 文档类型,如 sustechthesis,[]内是选项,如 degree=doctor
  % 这里开始是导言区
  \usepackage{graphicx} % 引用宏包
  \graphicspath{{fig/}} % 设置图片目录
  \def\rawcmd#1{\texttt{\color{DarkBlue}\footnotesize #1}}% 自定义新命令
  % 导言区到此为止
  \begin{document}
  这里开始是正文
  \end{document}\end{lstlisting}
\end{frame}

\subsection{常用命令}

\begin{frame}[fragile]{\LaTeX“命令”}
    \framesubtitle{\emph{宏} (Macro)、或者\emph{控制序列} (control sequence)}
    \begin{itemize}
        \item 简单命令
            \begin{itemize}
                \item \verb|\命令|\hspace{2em}
                    \verb|{\heiti 茴字的四种写法}| ~$\Rightarrow$ {\heiti 茴字的四种写法}
                \item \verb|\命令[可选参数]{必选参数}|\\
                    \verb|\section[精简标题]{这个题目实在太长了放到目录里面不太好看}|\\
                    $\Rightarrow$ {\heiti 1.1 \hspace{1em} \songti 这个题目实在太长了放到目录里面不太好看}
            \end{itemize}
        \item 环境命令
            \begin{columns}[c]
                \begin{column}{0.45\textwidth}
                    \begin{lstlisting}[basicstyle=\ttfamily]
\begin{equation*}
    a^2-b^2=(a+b)(a-b)
\end{equation*}\end{lstlisting}
                \end{column}
                \hspace{1em}
                \begin{column}{0.45\textwidth}
                    $ a^2-b^2=(a+b)(a-b)$
                \end{column}
            \end{columns}
    \end{itemize}
\end{frame}

% \begin{frame}[fragile]{\LaTeX{} 常用命令}
%   \begin{exampleblock}{简单命令}
% \centering
% \footnotesize
%   \begin{tabular}{llll}
%     \cmd{chapter} & \cmd{section} & \cmd{subsection} & \cmd{paragraph} \\
%     章 & 节 & 小节 & 带题头段落 \\\hline
%     \cmd{centering} & \cmd{emph} & \cmd{verb} & \cmd{url} \\
%    居中对齐         &  强调      & 原样输出   & 超链接 \\\hline
%   \cmd{footnote} & \cmd{item} & \cmd{caption} & \cmd{includegraphics} \\
%    脚注 & 列表条目 & 标题 & 插入图片 \\\hline
%   \cmd{label} & \cmd{cite} & \cmd{ref} \\
%   标号 & 引用参考文献 & 引用图表公式等\\\hline
%   \end{tabular}
% \end{exampleblock}
% \end{frame}
\begin{frame}[fragile]{谋篇布局}
    \begin{itemize}
        \item 一篇学位论文包括:
              \begin{itemize}
                  \item 标题:|\title|、|\author|、|\date| $\to$ |\maketitle|
                  \item 摘要:|abstract| 环境
                  \item 目录:|\tableofcontents|
                  \item 章节:|\chapter|、|\section|、|\subsection| 等
                  \item 图表:|table|、|figure|环境
                  \item 引用:|\label|、|\cite|、|\ref|
                  \item 文献:|\bibliography|
                  \item 附录:|\appendix|
                  \item 致谢:|acknowledgements| 环境
              \end{itemize}
        \item 文档划分
              \begin{itemize}
                  \item 页码划分:|\frontmatter|、|\mainmatter|、|\backmatter|
                  \item 分文件编译:|\include|、|\input|
              \end{itemize}
    \end{itemize}
\end{frame}

\begin{frame}[fragile]
    \frametitle{文本标记}
    \begin{itemize}
        \item 加粗:|{\bfseries ...}| 或 |\textbf{...}|
        \item 倾斜:|{\itshape ...}| 或 |\textit{...}|
        \item 字号:|\tiny|、|\small|、|\normalsize|、|\large|、|\huge| 等
        \item 换行:|\\|
        \item 缩进:|\indent|、|\noindent|
        \item 居中:|\centering| 或 |center| 环境
    \end{itemize}
\end{frame}


\begin{frame}{\LaTeX{} 命令举例}
    \cmdxmp{chapter}{前言}{\heiti 第 1 章\hspace{1em} 前言}
    \cmdxmp{section[精简标题]}{这个题目实在太长了放到目录里面不太好看}{\heiti 1.1
        \hspace{1em} 这个题目实在太长了放到目录里面不太好看}
    \cmdxmp{footnote}{我是可爱的脚注}{前方高能\footnotemark}
    \vspace{-40pt}
\end{frame}

\subsection{环境}
\begin{frame}[fragile]{\LaTeX{} 常用环境命令}
    \begin{itemize}
        \item \env{table}:用于创建一个表格环境
        \item \env{figure}:用于创建一个图片环境
        \item \env{itemize}:用于创建一个无编号列表,使用|\item|进行分点
        \item \env{enumerate}:用于创建一个编号列表,使用|\item|进行分点
        \item \env{equation}:用于创建一个公式环境,环境内适用行间公式语法
    \end{itemize}
\end{frame}

\subsection{列表}
\begin{frame}[fragile]{\LaTeX{} 环境举例}
    \vspace{1em}
    \begin{minipage}{0.5\linewidth}
        \begin{lstlisting}[basicstyle=\ttfamily\small]
\begin{itemize}
    \item 一条
    \item 次条
    \item 这一条可以分为 ...
    \begin{itemize}
        \item 子一条
    \end{itemize}
\end{itemize}\end{lstlisting}
    \end{minipage}\hspace{1.5cm}
    \begin{minipage}{0.3\linewidth}
        \(\bullet\)\quad 一条\\
        \(\bullet\)\quad 次条\\
        \(\bullet\)\quad 这一条可以分为 ... \\
        \quad - \quad 子一条
    \end{minipage}
    % \smallskip

    \begin{minipage}{0.5\linewidth}
        \begin{lstlisting}
\begin{enumerate}
    \item 一条
    \item 次条
    \item 再条
\end{enumerate}\end{lstlisting}
    \end{minipage}\hspace{1.5cm}
    \begin{minipage}{0.3\linewidth}
        \vspace{-1cm}
        1.\quad 一条\\
        2.\quad 次条\\
        3.\quad 再条
    \end{minipage}
\end{frame}
%

\begin{frame}[fragile]{列表与枚举}
    \begin{columns}
        \begin{column}{.6\textwidth}
            \begin{lstlisting}[basicstyle=\ttfamily\small]
\begin{enumerate}
\item \LaTeX{} 好处都有啥
    \begin{description}
    \item[好用:] 体验好才是真的好
    \item[好看:] 强迫症的福音
    \item[开源:] 众人拾柴火焰高
    \end{description}
\item 还有呢?
    \begin{itemize}
    \item 好处 1
    \item 好处 2
    \end{itemize}
\end{enumerate}\end{lstlisting}
        \end{column}
        \begin{column}{.4\textwidth}
            {\small
                1.\quad \LaTeX{} 好处都有啥\\
                \quad 好用: 体验好才是真的好\\
                \quad 好看: 治疗强迫症\\
                \quad 开源: 众人拾柴火焰高\\
                2.\quad 还有呢?\\
                \quad -\quad 好处 1\\
                \quad -\quad 好处 2
            }
        \end{column}
    \end{columns}

\end{frame}



\subsection{数学公式}
\begin{frame}[fragile]{\LaTeX{} 数学公式}
    \begin{itemize}
        \item 数学公式排版是 \LaTeX{} 的绝对强项
        \item 数学排版需要进入数学模式,引用 \texttt{amsmath} 宏包,由美国数学学会(American Mathematical Society, AMS) 提供。
              \begin{itemize}
                  \item 用单个美元符号(\verb|$|) (不推荐)或\verb|\( \)|包围起来的内容是{\bf 行内公式}
                  \item 用两个美元符号(\verb|$$|) (不推荐)或\verb|\[ \]|包围起来的是{\bf 单行公式}或{\bf 行间公式}
                  \item 使用数学环境,例如 \texttt{equation} 环境内的公式会自动加上编号,
                        \texttt{align} 环境用于多行公式(例如方程组、多个并列条件等)
              \end{itemize}
        \item 寻找符号
              \begin{itemize}
                  \item 运行 \texttt{texdoc symbols} 查看符号表
                  \item S. Pakin. \emph{The Comprehensive \LaTeX{} Symbol List}
                        \url{https://ctan.org/pkg/comprehensive}
                  \item 手写识别(有趣但不全):Detexify \url{http://detexify.kirelabs.org}
              \end{itemize}
        \item MathType 也可以使用和导出 \LaTeX{} 公式(不推荐)
        \item Mathpix Snip 识别图片导出
    \end{itemize}
\end{frame}


\begin{frame}[fragile]{\LaTeX{} 数学公式}

    \begin{columns}
        \begin{column}{.5\textwidth}
            \lstset{language=[LaTeX]TeX}
            \begin{lstlisting}[basicstyle=\ttfamily\small]
体积公式为:
\(V = \frac{4}{3}\pi r^3\)。

体积公式为:
\[
    V = \frac{4}{3}\pi r^3
\]

体积公式为:
\begin{equation}
\label{eq:vsphere}
V = \frac{4}{3}\pi r^3
\end{equation}\end{lstlisting}
        \end{column}
        \begin{column}{.5\textwidth}
            体积公式为:\(V = \frac{4}{3}\pi r^3\)。

            体积公式为:
            \[
                V = \frac{4}{3}\pi r^3
            \]

            体积公式为:
            \begin{equation}
                \label{eq:vsphere}
                V = \frac{4}{3}\pi r^3
            \end{equation}
        \end{column}
    \end{columns}

\end{frame}


\subsection{目录}
\begin{frame}[fragile]{层次与目录生成}
    \begin{columns}
        \begin{column}{.6\textwidth}
            \lstset{language=[LaTeX]TeX}
            \begin{lstlisting}[basicstyle=\ttfamily\small]
    \tableofcontents % 这里是目录
    \chapter{绪\quad 论}
    \section{研究工作的背景及意义}
    \section{国内外研究现状}
    \subsection{现有工作不足}
    \appendix
    \chapter{策略梯度公式推导}\end{lstlisting}
        \end{column}
        \begin{column}{.4\textwidth}
            \heiti
            {\centering 第一章\quad 绪\quad 论 \\}
            1.1 研究工作的背景及意义 \\
            1.2 国内外研究现状 \\
            1.2.1 现有工作不足 \\
            {\centering 附录 A\quad 策略梯度公式推导\\}
        \end{column}
    \end{columns}

\end{frame}

\subsection{插图,表格,交叉引用}
\begin{frame}[fragile]{交叉引用与插入插图}
    \begin{itemize}
        \item 给对象命名:图片、表格、公式等\\
              |\label{name}|
        \item 引用对象\\
              |\ref{name}|
    \end{itemize}
    \bigskip

    \begin{minipage}{0.7\linewidth}
        \lstset{language=[LaTeX]TeX}
        \begin{lstlisting}
\begin{figure}[htbp]
\centering
\includegraphics[height=.2\textheight]{LOGO.png}
\caption{南科大校徽。}
\label{fig:sustech:LOGO}
\end{figure}
南科大校徽请参见图~\ref{fig:sustech:LOGO}。\end{lstlisting}
    \end{minipage}\hfill
    \begin{minipage}{0.3\linewidth}\centering
        \includegraphics[height=0.2\textheight]{LOGO.png}\\
        {\footnotesize\heiti 图~1. 南科大校徽。}\\
        {\songti 南科大校徽请参见图~1。}\\[1em]
    \end{minipage}
\end{frame}

\begin{frame}[fragile]{交叉引用与插入表格}
    \begin{columns}
        \column{.6\textwidth}
        \lstset{language=[LaTeX]TeX}
        \begin{lstlisting}
\begin{table}[htbp]
    \caption{编号与含义}
    \label{tab:number}
    \centering
    \begin{tabular}{cl}
    \hline
    编号 & 含义 \\
    \hline
    1    & 第一 \\
    2    & 第二 \\
    \hline
    \end{tabular}
\end{table}
公式~(\ref{eq:vsphere})中编号与含义请参见
表~\ref{tab:number}。\end{lstlisting}
        \column{.4\textwidth}
        \centering
        {\small 表~1. 编号与含义}\\[2pt]
        \begin{tabular}{cl}\hline
            编号 & 含义 \\\hline
            1  & 第一 \\
            2  & 第二 \\\hline
        \end{tabular}\\[5pt]

        \normalsize 公式~(\ref{eq:vsphere})编号与含义请参见表~1。
    \end{columns}
\end{frame}

\begin{frame}[fragile]{浮动体}
    \begin{itemize}
        \item 初学者最“捉摸不透”的特性之一 \url{https://liam.page/2017/03/11/floats-in-LaTeX-basic}
        \item 图片和表格有时会很大,在插入的位置不一定放得下,因此需要浮动调整
        \item 避免在文中使用|下图||下图|的说法,而是使用图表的编号,例如 |图~\ref{fig:fig1}|
        \item 一般来讲,在写论文时,应尽可能保证浮动体(尤其是图片)出现在引用文字前
        \item |\begin{figure}[<位置参数>] 图片 \end{figure}|
              \begin{itemize}
                  \item 位置参数指定浮动体摆放的偏好
                  \item |h| 当前位置(here), |t| 顶部(top), |b| 底部(bottom), |p| 单独成页(p)
                  \item |!h| 表示忽略一些限制,|H| 表示强制\alert{(强烈不建议,除非你知道自己在做什么)}
              \end{itemize}
        \item 必要时,可以通过在段落中间插入浮动体的方式,来使得页面上下不留出太多空白。
    \end{itemize}
\end{frame}

\begin{frame}[fragile]
    \frametitle{作图与插图}
    \begin{itemize}
        \item 外部插入
              \begin{itemize}
                  \item Mathematica、MATLAB
                  \item PowerPoint、Visio、Adobe Illustrator、Inkscape
                  \item Python \pkg{Matplotlib} 库、\texttt{Plots.jl}、R、Plotly 等
                  \item draw.io \url{https://draw.io/}、ProcessOn \url{https://www.processon.com/} 等在线绘图网站
              \end{itemize}
        \item \TeX 内联
              \begin{itemize}
                  \item Asymptote
                  \item \alert{\pkg{pgf}/\pkg{TikZ}、\pkg{pgfplots}}
              \end{itemize}

        \item 插图格式

              \begin{itemize}
                  \item 矢量图:|.pdf| 或 |.eps|
                  \item 位图:|.jpg| 或 |.png|
                  \item 不(完全)支持 |.svg|、|.bmp|
              \end{itemize}

        \item 参考:如何在论文中画出漂亮的插图?\link{https://www.zhihu.com/question/21664179}
    \end{itemize}
\end{frame}

\begin{frame}[fragile]{表格绘制}
    \begin{itemize}
        \item 使用 \pkg{booktabs}(三线表)、\pkg{longtables}(跨页表)、\pkg{multirow}(单元格内换行) 等宏包
        \item 手动绘制表格确实比较令人头疼,且较难维护
        \item 推荐使用在线工具绘制后导出代码:
              \begin{itemize}
                  \item \LaTeX{} Tables Editor \link{https://www.latex-tables.com/}
                  \item \LaTeX{} Table Generator \link{https://www.tablesgenerator.com/latex_tables}
              \end{itemize}
        \item 使用Excel插件:excel2latex \link{https://www.ctan.org/tex-archive/support/excel2latex/}
        \item 想要更加丰富的表格列宽控制和文本对齐功能?试试\pkg{tabularx}
        \item 命令太复杂,记不住!
              \begin{itemize}
                  \item 请仔细阅读文档
                  \item 当然,你可以直接向ChatGPT提问,但请描述清楚你的需求,并确认它给出的方案是可行且正确的。
              \end{itemize}
    \end{itemize}
\end{frame}



\subsection{文献管理}
\begin{frame}[fragile]
    \frametitle{文献管理}
    \begin{itemize}
        \item 建议自动生成\pause (你只有三篇参考文献?)\pause
        \item |.bib| 数据库
              \begin{itemize}
                  \item Google Scholar 可直接复制:点击 \faQuoteRight \quad -> BibTeX
                  \item 用 EndNote、Jabref 等生成
                  \item \alert{一定注意校对!!!}
              \end{itemize} \pause
        \item 传统方法(大部分会议、期刊模板):\BibTeX  后端
              \begin{itemize}
                  \item 控制文献、引用样式:\pkg{natbib} 宏包
                  \item 国家标准 GB/T 7714--2015
                        \link{https://www.gb688.cn/bzgk/gb/newGbInfo?hcno=7FA63E9BBA56E60471AEDAEBDE44B14C}
                        \link{https://github.com/Haixing-Hu/GBT7714-2005-BibTeX-Style/files/153951/GBT.7714-2015.pdf}:
                        \alert{\pkg{gbt7714} 宏包}
              \end{itemize} \pause
        \item 现代方法:\pkg{biber} 后端 + \pkg{biblatex} 宏包

              \begin{itemize}
                  \item 国家标准:\pkg{biblatex-gb7714-2015} 宏包
              \end{itemize} \pause

        \item 需多次编译
              \begin{itemize}
                  \item \pdfLaTeX -> \BibTeX -> \pdfLaTeX -> \pdfLaTeX
                  \item \XeLaTeX -> \BibTeX -> \XeLaTeX -> \XeLaTeX
                  \item 一键使用:\pkg{VS Code plugin}, \pkg{MakeFile}, \pkg{Batch} script, \pkg{latexmk}
              \end{itemize}

    \end{itemize}
\end{frame}

\begin{frame}[fragile]{引用样例}
    \lstset{language=[LaTeX]TeX}
    \begin{columns}
        \begin{column}{.6\textwidth}
            \begin{lstlisting}[basicstyle=\ttfamily\small]
% In body.tex
“真理只有一个,而究竟谁发现了真理,不依靠主观的夸张,而依靠客观的实践。”-- 毛泽东\cite{毛泽东1949新民主主义论}。

% In references.bib
@book{毛泽东1949新民主主义论,
  title={新民主主义论},
  author={毛泽东},
  year={1949},
  publisher={长江出版社}
}
    \end{lstlisting}
        \end{column}
        \begin{column}{.4\textwidth}
            “真理只有一个,而究竟谁发现了真理,不依靠主观的夸张,而依靠客观的实践。” -- 毛泽东\cite{毛泽东1949新民主主义论}。

            \printbibliography[heading=none]
        \end{column}
    \end{columns}
\end{frame}

% \begin{frame}[fragile]
%   \frametitle{测试}
%   \begin{lstlisting} code \end{lstlisting}

%     % \begin{minipage}{0.4\linewidth}
%     %   \begin{lstlisting}[basicstyle=\ttfamily\small]
%     %     “真理只有一个,而究竟谁发现了真理,不依靠主观的夸张,而依靠客观的实践。” -- 毛泽东\cite{毛泽东1949新民主主义论}。\end{lstlisting}
%     % \end{minipage}
% %     \begin{minipage}{0.4\linewidth}
% %       \begin{lstlisting}[basicstyle=\ttfamily\small]
% %         @book{毛泽东1949新民主主义论,
% %   title={新民主主义论},
% %   author={毛泽东},
% %   year={1949},
% %   publisher={长江出版社}
% % }\end{lstlisting}
% %     \end{minipage}
% %     \begin{minipage}{0.4\linewidth}
% % \begin{lstlisting}
% %   “真理只有一个,而究竟谁发现了真理,不依靠主观的夸张,而依靠客观的实践。” -- 毛泽东\cite{毛泽东1949新民主主义论}。
% % \end{lstlisting}

% % “真理只有一个,而究竟谁发现了真理,不依靠主观的夸张,而依靠客观的实践。” -- 毛泽东\cite{毛泽东1949新民主主义论}。
% %     \end{minipage}\hspace{1.5cm}
% %     \begin{minipage}{0.4\linewidth}
% %       \printbibliography
% %     \end{minipage}\hspace{1.5cm}
% \end{frame}
